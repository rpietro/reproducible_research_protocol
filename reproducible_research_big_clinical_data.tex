%%%%%%%%%%%%%%%%%%%%%%%%%%%%%%%%%%%%%%%%%
% Journal Article
% LaTeX Template
% Version 1.1 (25/11/12)
%
% This template has been downloaded from:
% http://www.LaTeXTemplates.com
%
% Original author:
% Frits Wenneker (http://www.howtotex.com)
%
% License:
% CC BY-NC-SA 3.0 (http://creativecommons.org/licenses/by-nc-sa/3.0/)
%
%%%%%%%%%%%%%%%%%%%%%%%%%%%%%%%%%%%%%%%%%

%----------------------------------------------------------------------------------------
%	PACKAGES AND OTHER DOCUMENT CONFIGURATIONS
%----------------------------------------------------------------------------------------

\documentclass[twoside]{article}

\usepackage{lipsum} % Package to generate dummy text throughout this template

\usepackage[sc]{mathpazo} % Use the Palatino font
\usepackage[T1]{fontenc} % Use 8-bit encoding that has 256 glyphs
\linespread{1.05} % Line spacing - Palatino needs more space between lines
\usepackage{microtype} % Slightly tweak font spacing for aesthetics

\usepackage[hmarginratio=1:1,top=32mm,columnsep=20pt]{geometry} % Document margins
\usepackage{multicol} % Used for the two-column layout of the document
\usepackage{hyperref} % For hyperlinks in the PDF

\usepackage[hang, small,labelfont=bf,up,textfont=it,up]{caption} % Custom captions under/above floats in tables or figures
\usepackage{booktabs} % Horizontal rules in tables
\usepackage{float} % Required for tables and figures in the multi-column environment - they need to be placed in specific locations with the [H] (e.g. \begin{table}[H])

\usepackage{lettrine} % The lettrine is the first enlarged letter at the beginning of the text
\usepackage{paralist} % Used for the compactitem environment which makes bullet points with less space between them

\usepackage{abstract} % Allows abstract customization
\renewcommand{\abstractnamefont}{\normalfont\bfseries} % Set the "Abstract" text to bold
\renewcommand{\abstracttextfont}{\normalfont\small\itshape} % Set the abstract itself to small italic text

\usepackage{titlesec} % Allows customization of titles
\renewcommand\thesection{\Roman{section}}
\titleformat{\section}[block]{\large\scshape\centering}{\thesection.}{1em}{} % Change the look of the section titles

\usepackage{fancyhdr} % Headers and footers
\pagestyle{fancy} % All pages have headers and footers
\fancyhead{} % Blank out the default header
\fancyfoot{} % Blank out the default footer
\fancyhead[C]{Running title $\bullet$ November 2012 $\bullet$ Vol. XXI, No. 1} % Custom header text
\fancyfoot[RO,LE]{\thepage} % Custom footer text

%----------------------------------------------------------------------------------------
%	TITLE SECTION
%----------------------------------------------------------------------------------------

\title{\vspace{-15mm}\fontsize{24pt}{10pt}\selectfont\textbf{Reproducible Research applied to Big Clinical Data - a standard operating procedure}} % Article title

\author{
\large
\textsc{Joao Vissoci}\\[2mm] % Your name
\normalsize Duke University Medical Center \\ % Your institution
\normalsize \href{mailto:jnv4@duke.edu}{jnv4@duke.edu} \\\\% Your email address
\textsc{Name Last Name}\\[2mm] % Your name
\normalsize Duke University Medical Center \\ % Your institution
\normalsize \href{mailto:email@duke.edu}{email@duke.edu} \\\\% Your email address
\textsc{Ricardo Pietrobon}\\[2mm] % Your name
\normalsize Duke University  Medical Center\\ % Your institution
\normalsize \href{mailto:rpietro@duke.edu}{rpietro@duke.edu} \\% Your email address
\vspace{-5mm}
}

\date{}

%----------------------------------------------------------------------------------------

\begin{document}

\maketitle % Insert title

\thispagestyle{fancy} % All pages have headers and footers

%----------------------------------------------------------------------------------------
%	ABSTRACT
%----------------------------------------------------------------------------------------

\begin{abstract}

Will write at the end

\end{abstract}

%----------------------------------------------------------------------------------------
%	ARTICLE CONTENTS
%----------------------------------------------------------------------------------------

\begin{multicols}{2} % Two-column layout throughout the main article text

\section{Introduction}

\lettrine[nindent=0em,lines=3]{W} ith the growing number of large healthcare data sets, the volume of scientific publications attempting to convert these data into clinically useful information is significantly increasing. And yet, with the increasing complexity in the data management, modeling and communication of results, the likelihood of the final information not being correct is also increased. As a result, a number of investigators is now focused on developing reproducible research protocols that would allow for the analysis of large data sets to be entirely reproducible, meaning that the results reported in a scientific publication could be immediately generated by having access to both data sets as well as the statistical and data mining scripts generating those results. 

\begin{compactitem}
\item lit review on reproducible research - include papers by heather piowar, CRAN taskview on reproducible research
\end{compactitem}

The objective of this study is to present a simple reporting framework for reproducible research applied to Big Clinical Data.


%------------------------------------------------

\section{Reproducible Research Reporting Framework}
%Joao, main idea here is to (1) describe what they are, and (2) how we use them in practice. 

\subsection{Data formats}

\subsubsection{CSV} 

\subsubsection{RDF, LOD and SPARQL endpoints}

\subsubsection{JSON}

\subsection{Data repositories}

\subsubsection{Figshare}

\subsubsection{Dryad}

\subsubsection{Google drive}
\href{http://googleappsdeveloper.blogspot.com/2012/11/announcing-google-drive-site-publishing.html}{Google Drive Site Publishing
}

\subsubsection{Github}

%------------------------------------------------


\subsection{Analytical scripts}

\subsubsection{R}
Glue for other languages and technologies such as Python, Java, relational databases, RDF, C, C++, Weka, among many others

\subsubsection{Reproducible research taskview}
knitr vs. sweave
need better ways to format tables

%------------------------------------------------


\subsection{Licensing}
Creative Commons

%------------------------------------------------
\subsection{Overall workflow}
%Joao, perhaps we could include a picture here showing the whole workflow

\begin{table}[H]
\caption{Example table}
\centering
\begin{tabular}{llr}
\toprule
\multicolumn{2}{c}{Name} \\
\cmidrule(r){1-2}
First name & Last Name & Grade \\
\midrule
John & Doe & $7.5$ \\
Richard & Miles & $2$ \\
\bottomrule
\end{tabular}
\end{table}


%------------------------------------------------

\section{Discussion}

\lipsum[7-8] % Dummy text

%----------------------------------------------------------------------------------------
%	REFERENCE LIST
%----------------------------------------------------------------------------------------

\begin{thebibliography}{99} % Bibliography - this is intentionally simple in this template

\bibitem[Figueredo and Wolf, 2009]{Figueredo:2009dg}
Figueredo, A.~J. and Wolf, P. S.~A. (2009).
\newblock Assortative pairing and life history strategy - a cross-cultural
  study.
\newblock {\em Human Nature}, 20:317--330.
 
\end{thebibliography}

%----------------------------------------------------------------------------------------

\end{multicols}

\end{document}
