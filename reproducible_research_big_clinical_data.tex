%%%%%%%%%%%%%%%%%%%%%%%%%%%%%%%%%%%%%%%%%
% Journal Article
% LaTeX Template
% Version 1.1 (25/11/12)
%
% This template has been downloaded from:
% http://www.LaTeXTemplates.com
%
% Original author:
% Frits Wenneker (http://www.howtotex.com)
%
% License:
% CC BY-NC-SA 3.0 (http://creativecommons.org/licenses/by-nc-sa/3.0/)
%
%%%%%%%%%%%%%%%%%%%%%%%%%%%%%%%%%%%%%%%%%

%----------------------------------------------------------------------------------------
%	PACKAGES AND OTHER DOCUMENT CONFIGURATIONS
%----------------------------------------------------------------------------------------

\documentclass[twoside]{article}

 \usepackage{acronym} % package to generate accronyms

\usepackage[sc]{mathpazo} % Use the Palatino font
\usepackage[T1]{fontenc} % Use 8-bit encoding that has 256 glyphs
\linespread{1.05} % Line spacing - Palatino needs more space between lines
\usepackage{microtype} % Slightly tweak font spacing for aesthetics

\usepackage[hmarginratio=1:1,top=32mm,columnsep=20pt]{geometry} % Document margins
\usepackage{multicol} % Used for the two-column layout of the document
\usepackage{hyperref} % For hyperlinks in the PDF

\usepackage[hang, small,labelfont=bf,up,textfont=it,up]{caption} % Custom captions under/above floats in tables or figures
\usepackage{booktabs} % Horizontal rules in tables
\usepackage{float} % Required for tables and figures in the multi-column environment - they need to be placed in specific locations with the [H] (e.g. \begin{table}[H])

\usepackage{lettrine} % The lettrine is the first enlarged letter at the beginning of the text
\usepackage{paralist} % Used for the compactitem environment which makes bullet points with less space between them

\usepackage{abstract} % Allows abstract customization
\renewcommand{\abstractnamefont}{\normalfont\bfseries} % Set the "Abstract" text to bold
\renewcommand{\abstracttextfont}{\normalfont\small\itshape} % Set the abstract itself to small italic text

\usepackage{titlesec} % Allows customization of titles
\renewcommand\thesection{\Roman{section}}
\titleformat{\section}[block]{\large\scshape\centering}{\thesection.}{1em}{} % Change the look of the section titles

\usepackage{fancyhdr} % Headers and footers
\pagestyle{fancy} % All pages have headers and footers
\fancyhead{} % Blank out the default header
\fancyfoot{} % Blank out the default footer
\fancyhead[C]{Running title $\bullet$ November 2012 $\bullet$ Vol. XXI, No. 1} % Custom header text
\fancyfoot[RO,LE]{\thepage} % Custom footer text

%----------------------------------------------------------------------------------------
%	TITLE SECTION
%----------------------------------------------------------------------------------------

\title{\vspace{-15mm}\fontsize{24pt}{10pt}\selectfont\textbf{A Framework for Reproducible, Interactive Research applied to Big Clinical Data}} % Article title

\author{
\large
\textsc{Joao Vissoci}\\[2mm] % Your name
\normalsize Duke University Medical Center \\ % Your institution
\normalsize \href{mailto:jnv4@duke.edu}{jnv4@duke.edu} \\\\% Your email address
\textsc{Name Last Name}\\[2mm] % Your name
\normalsize Duke University Medical Center \\ % Your institution
\normalsize \href{mailto:email@duke.edu}{email@duke.edu} \\\\% Your email address
\textsc{Ricardo Pietrobon}\\[2mm] % Your name
\normalsize Duke University  Medical Center\\ % Your institution
\normalsize \href{mailto:rpietro@duke.edu}{rpietro@duke.edu} \\% Your email address
\vspace{-5mm}
}

\date{}

%----------------------------------------------------------------------------------------

\begin{document}

\maketitle % Insert title

\thispagestyle{fancy} % All pages have headers and footers

%----------------------------------------------------------------------------------------
%	ABSTRACT
%----------------------------------------------------------------------------------------

\begin{abstract}

Will write at the end

\end{abstract}

%----------------------------------------------------------------------------------------
%	ARTICLE CONTENTS
%----------------------------------------------------------------------------------------

\begin{multicols}{2} % Two-column layout throughout the main article text

\section{Introduction}

\lettrine[nindent=0em,lines=3]{W} ith the growing number of large healthcare data sets, the volume of scientific publications attempting to convert these data into clinically useful information is significantly increasing. 
And yet, with the increasing complexity in the data management, modeling and communication of results, the likelihood of the final information not being correct is also increased. 
As a result, a number of investigators is now focused on developing reproducible research protocols that would allow for the analysis of large data sets to be entirely reproducible, meaning that the results reported in a scientific publication could be immediately generated by having access to both data sets as well as the statistical and data mining scripts generating those results. 

lit review on reproducible research - include papers by heather piowar, CRAN taskview on reproducible research

The objective of this study is to present a simple reporting framework for reproducible, interactive research applied to Big Clinical Data.


%------------------------------------------------

\section{Reproducible Research Framework}
%Joao, main idea here is to (1) describe what they are, and (2) how we use them in practice. 

\subsection{Data formats}

\subsubsection{CSV, comma separated values} 
 
Comma separated values (CSV) is a format that is readily available for consumption by virtually any data analysis language or software. 
Although it does not provide a way for the downloaded data to be regularly updated other than downloading the data set again, also not offering any security features, csv files have one the best usability experiences among all formats. % tried \acronym{CSV} {Comma Separated Values}  but it is not working, need to find another package

Ways of enhancing the ability  to update and secure csv files include %Jose Eduardo, what would you suggest?

\subsubsection{RDF, LOD and SPARQL endpoints}
Web semantic technologies have recently become popular given the success provided by Linked Open Data (LOD). The framework is referred to as Resource Description Framework (RDF), while data sets themselves are queried through the SPARQL (a recursive acronym for SPARQL Protocol and RDF Query Language). Main advantages include the data availability 24/7, with automated updates, but also the ability to dynamically merge across data sets sharing identical elements (classes), also allowing for security features. %Joao, in the security section we should talk about http://goo.gl/Okclf

\subsubsection{JSON}

\subsection{Data repositories}

\subsubsection{Figshare}

\subsubsection{Dryad}

\subsubsection{Google drive}
\href{http://googleappsdeveloper.blogspot.com/2012/11/announcing-google-drive-site-publishing.html}{Google Drive Site Publishing
}

\subsubsection{Github}

%------------------------------------------------


\subsection{Analytical scripts}

\subsubsection{R}
Glue for other languages and technologies such as Python, Java, relational databases, RDF, C, C++, Weka, among many others

\subsubsection{Reproducible research taskview}
knitr vs. sweave
need better ways to format tables

%------------------------------------------------


\subsection{Dynamic research}

Joao, below is copied and pasted from http://goo.gl/QG61b - you need to select which packages you might want to use

Interactive Graphics : There are several efforts to implement interactive graphics systems that interface well with R. In an interactive system the user can interactively query the graphics on the screen with the mouse, or a moveable brush to zoom, pan and query on the device as well as link with other views of the data. rggobi embeds the GGobi interactive graphics system within R, so that one can display a data frame or several in GGobi directly from R. The package has functions to support longitudinal data, and graphs using GGobi's edge set functionality. The RoSuDA repository maintained and developed by the University of Augsburg group has two packages, iplots and iwidgets as well as their Java development environment including a Java device, JavaGD. Their interactive graphics tools contain functions for alpha blending, which produces darker shading around areas with more data. This is exceptionally useful for parallel coordinate plots where many lines can quickly obscure patterns. playwith has facilities for building interactive versions of R graphics using the cairoDevice and RGtk2. Lastly, the rgl package has mechanisms for interactive manipulation of plots, especially 3D rotations and surfaces.
%------------------------------------------------


\subsection{Licensing}
Creative Commons

%------------------------------------------------
\subsection{Overall workflow}
%Joao, perhaps we could include a picture here showing the whole workflow

\begin{table}[H]
\caption{Example table}
\centering
\begin{tabular}{llr}
\toprule
\multicolumn{2}{c}{Name} \\
\cmidrule(r){1-2}
First name & Last Name & Grade \\
\midrule
John & Doe & $7.5$ \\
Richard & Miles & $2$ \\
\bottomrule
\end{tabular}
\end{table}


%------------------------------------------------

\section{Discussion}

\lipsum[7-8] % Dummy text

%----------------------------------------------------------------------------------------
%	REFERENCE LIST
%----------------------------------------------------------------------------------------

\begin{thebibliography}{99} % Bibliography - this is intentionally simple in this template

\bibitem[Figueredo and Wolf, 2009]{Figueredo:2009dg}
Figueredo, A.~J. and Wolf, P. S.~A. (2009).
\newblock Assortative pairing and life history strategy - a cross-cultural
  study.
\newblock {\em Human Nature}, 20:317--330.
 
\end{thebibliography}

%----------------------------------------------------------------------------------------

\end{multicols}

\end{document}
